%: CLASS FILE

\documentclass[11pt]{article}
%\documentclass[11pt,reqno]{amsart} % AMS article style with right equation numbering


%: PACKAGES

\usepackage{graphicx}

% sets margins
\usepackage[top=1in,bottom=1in,left=1in,right=1in]{geometry}

% gives more color definitions
\usepackage{xcolor}

% gives optional arguments to list environments
\usepackage{enumitem}

\usepackage{microtype}

% uncomment to begin paragraphs with an empty line rather than an indent
% \usepackage[parfill]{parskip}

\usepackage{amsmath, amsthm, amssymb, tikz, mathtools, array, mathrsfs, tensor}

% must be loaded after ams packages
% gives various integral symbols: see http://ctan.math.illinois.edu/macros/latex/contrib/esint/esint.pdf
\usepackage{esint}

% creates command \one for blackboard 1 to use for indicator functions
\usepackage{dsfont}
	\newcommand{\one}{\mathds{1}}

% \framebox{$ stuff $} will frame things in line.
% \Aboxed{$ stuff $} will frame things in align environment.
\usepackage{framed}

% always have this as the last package listed.
\usepackage{hyperref}
	\hypersetup{hypertexnames=false,colorlinks=true,linkcolor=blue,citecolor=black,urlcolor=blue}
	
	
%: SETTINGS

% uncomment to align environments to be split over page breaks
%\allowdisplaybreaks

%: COMMANDS

% equation shortcuts
\newcommand{\eq}[1]{\begin{align*} #1 \end{align*}}
\newcommand{\eeq}[1]{\begin{align} \begin{split} #1 \end{split} \end{align}}

% shortcut for stacking equation references over relations
% usage: \stackref{LABEL}{SYMBOL}
\newcommand{\stackref}[2]{\stackrel{\mbox{\footnotesize{\eqref{#1}}}}{#2}}
\newcommand{\stackrefp}[2]{\stackrel{\phantom{\mbox{\footnotesize{\eqref{#1}}}}}{#2}}

% variable Greek letters
\def\eps{\varepsilon}
\def\vphi{\varphi}

% blackboard letters
\newcommand{\E}{\mathbb{E}}
\newcommand{\F}{\mathbb{F}}
\newcommand{\N}{\mathbb{N}}
\renewcommand{\P}{\mathbb{P}}
\newcommand{\Q}{\mathbb{Q}}
\newcommand{\R}{\mathbb{R}}
\renewcommand{\S}{\mathbb{S}}
\newcommand{\T}{\mathbb{T}}
\newcommand{\Z}{\mathbb{Z}}

% calligraphic letters
\renewcommand{\AA}{\mathcal{A}}
\newcommand{\BB}{\mathcal{B}}
\newcommand{\CC}{\mathcal{C}}
\newcommand{\DD}{\mathcal{D}}
\newcommand{\EE}{\mathcal{E}}
\newcommand{\FF}{\mathcal{F}}
\newcommand{\GG}{\mathcal{G}}
\newcommand{\HH}{\mathcal{H}}
\newcommand{\II}{\mathcal{I}}
\newcommand{\JJ}{\mathcal{J}}
\newcommand{\KK}{\mathcal{K}}
\newcommand{\LL}{\mathcal{L}}
\newcommand{\MM}{\mathcal{M}}
\newcommand{\NN}{\mathcal{N}}
\newcommand{\OO}{\mathcal{O}}
\newcommand{\PP}{\mathcal{P}}
\newcommand{\QQ}{\mathcal{Q}}
\newcommand{\RR}{\mathcal{R}}
\renewcommand{\SS}{\mathcal{S}}
\newcommand{\TT}{\mathcal{T}}
\newcommand{\UU}{\mathcal{U}}
\newcommand{\VV}{\mathcal{V}}
\newcommand{\WW}{\mathcal{W}}
\newcommand{\XX}{\mathcal{X}}
\newcommand{\YY}{\mathcal{Y}}
\newcommand{\ZZ}{\mathcal{Z}}

% script letters
\newcommand{\AAA}{\mathcal{A}}
\newcommand{\BBB}{\mathcal{B}}
\newcommand{\CCC}{\mathcal{C}}
\newcommand{\DDD}{\mathcal{D}}
\newcommand{\EEE}{\mathcal{E}}
\newcommand{\FFF}{\mathcal{F}}
\newcommand{\GGG}{\mathcal{G}}
\newcommand{\HHH}{\mathcal{H}}
\newcommand{\III}{\mathcal{I}}
\newcommand{\JJJ}{\mathcal{J}}
\newcommand{\KKK}{\mathcal{K}}
\newcommand{\LLL}{\mathcal{L}}
\newcommand{\MMM}{\mathcal{M}}
\newcommand{\NNN}{\mathcal{N}}
\newcommand{\OOO}{\mathcal{O}}
\newcommand{\PPP}{\mathcal{P}}
\newcommand{\QQQ}{\mathcal{Q}}
\newcommand{\RRR}{\mathcal{R}}
\newcommand{\SSS}{\mathcal{S}}
\newcommand{\TTT}{\mathcal{T}}
\newcommand{\UUU}{\mathcal{U}}
\newcommand{\VVV}{\mathcal{V}}
\newcommand{\WWW}{\mathcal{W}}
\newcommand{\XXX}{\mathcal{X}}
\newcommand{\YYY}{\mathcal{Y}}
\newcommand{\ZZZ}{\mathcal{Z}}

% inner product 
% usage: \iprod{left}{right}
\newcommand{\iprod}[2]{\langle #1,\, #2\rangle}

% bold math font (still in italics) 
% usage: {\vc CONTENT}
\newcommand{\vc}[1]{{\boldsymbol #1}}

% widetilde and widehat shortcuts
\newcommand{\wt}[1]{\widetilde{#1}}
\newcommand{\wh}[1]{\widehat{#1}}

% math operators with subcripts
\DeclareMathOperator{\Var}{Var}
\DeclareMathOperator{\Cov}{Cov}
\DeclareMathOperator{\Ent}{Ent}

% math operators with subtext
\DeclareMathOperator*{\esssup}{ess\,sup}
\DeclareMathOperator*{\essinf}{ess\,inf}

% conditional statements (size of delimiter as an optional command)
\newcommand{\givenc}[3][]{#1\{ #2 \: #1| \: #3 #1\}} % with braces
\newcommand{\givenk}[3][]{#1[ #2 \: #1| \: #3 #1]} % with brackets
\newcommand{\givenp}[3][]{#1( #2 \: #1| \: #3 #1)} % with parentheses
\newcommand{\givena}[3][]{#1\langle #2 \: #1| \: #3 #1\rangle} % with angle brackets

% common math text
\DeclareMathOperator{\e}{e} % exponential
\newcommand{\cc}{\mathrm{c}} % for set complements and "critical" subscripts
\newcommand{\dd}{\mathrm{d}} % for differentials
\newcommand{\Exp}{\mathrm{Exp}} % for exponential distribution

% floor and ceiling functions
\DeclarePairedDelimiter\ceil{\lceil}{\rceil}
\DeclarePairedDelimiter\floor{\lfloor}{\rfloor}

%: ENVIRONMENTS

% exercise environment
\newenvironment{exercise}[1]
	{\noindent \textbf{#1:}}
	{\par \vspace{0.5\baselineskip}}
	
% solution environment
\newenvironment{solution}[1][\unskip]
	{\noindent \textbf{Solution #1:} }
	{\qed \pagebreak}
	
\newenvironment{solutionn}[1][\unskip]
	{\noindent \textbf{Solution #1:} }
	{\qed \\}	


%: TITLE

\begin{document}
\bibliographystyle{acm}

\begin{center}
	\framebox{\parbox{\linewidth}{\centering
	{\bf{Homework 4}}\\
	MATH 541: Abstract Algebra 1\\
	Spring 2023 \\[\baselineskip]
	{\sc Hongtao Zhang}}} %at \href{mailto:ewbates@wisc.edu}{\nolinkurl{ewbates@wisc.edu}}.}}
\end{center}


%: BODY

Sec. 3.1: 2, 14, 22, 24, 36, 40, 41

\section*{2}

\textsf{Let \(\varphi: G \to H\) be a homomorphism of groups with kernel \(K\) and let \(a, b \in \varphi(G)\). Let \(X \in G/K\) be the fiber above \(a\) and let \(Y\) be the fiber above \(b\), i.e., \(X = \varphi^{-1}(a), Y = \varphi^{-1}(b)\). Fix an element \(u\) of \(X\) (so \(\varphi(u) = a\)). Prove that if \(XY = Z\) in the quotient group \(G/K\) and \(w\) is any member of \(Z\), then there is some \(v \in Y\) such that \(uv = w\). [Show \(u^{-1}w \in Y\)]}



\begin{proof}
	\[
		Z=XY=\phi^{-1}(a)\phi^{-1}(b) = \phi^{-1}(ab)
	\]

	We know that $\phi$ is a group homomorphism, so $\phi(xy)=\phi(x)\phi(y)$.

	Therefore, $\phi(u^{-1}w) = \phi(u^{-1})\phi(w) = a^{-1} ab=b$, which implies that $u^{-1}w \in Y$.
\end{proof}

\begin{exercise}{14}
	\textsf{Consider the additive quotient group \(\mathbb{Q}/\mathbb{Z}\)}
	\begin{enumerate}[label=\emph{\alph*}]
		\item \textsf{Show that every coset of \(\mathbb{Z}\) in \(\mathbb{Q}\) contains exactly one representative \(q \in \mathbb{Q}\) in the range \(0 \leqslant q < 1\).}
		\item \textsf{Show that every element of \(\mathbb{Q}/\mathbb{Z}\) has finite order but that there are elements of arbitrarily large order.}
		\item \textsf{Show that \(\mathbb{Q}/\mathbb{Z}\) is the torsion subgroup of \(\mathbb{R}/\mathbb{Z}\) (cf. Exercise 6, Section 2.1).}
		\item \textsf{Show that \(\mathbb{Q}/\mathbb{Z}\) is isomorphic to the multiplicative group of root of unity in \(\mathbb{C}^{\times}\).}
	\end{enumerate}
\end{exercise}{22}

\begin{proof}
	\begin{enumerate}[label=\emph{\alph*}]
		\item \[
			      \forall q' \in \Q: q'\Z = \{q'+z : z \in \Z\}, \exists z' \in \Z: q'+z' \in [0,1), q = q'+z' \implies q\Z = (q'+z')\Z
		      \]
		      It is easy to see that $(q'+z')\Z = (q')\Z$, which means $q\Z=q'\Z$
		\item Note $\Q/\Z$ is an equivalence classes based on all $q \in [0,1)$.
		      \[
			      \forall q \in \Q/\Z, \exists z_1,z_2 \in \Z: q = \frac{z_1}{z_2}
		      \]

		      Therefore, the order is $(\operatorname{lcm} (z_1,z_2))/z_1$, which can be arbitrarily large but finite.

		\item It suffices to show $\Q/\Z$ is a subgroup $\R/\Z$ with question $b$.
		      We know that the sum of two rational number is a rational number.
		      We also know that the inverse of a rational number is $-q$ which is equivalent to $1-q$.
		      $0$ is clearly a rational number.
		\item We can just write out the isomorphism.

		      For all order $z \in \Z$, the root of unity contains exactly $z$ elements such that $z$ is the order of the element.

		      Therefore, we can just map all element from $\frac{\Q}{\Z}$ to the root of unity with the denominator as the order,
		      and numerator as the index of the element, which is bijective, and vice versa.

	\end{enumerate}
\end{proof}

\begin{exercise}{22}
	\begin{enumerate}[label=\emph{\alph*}]
		\item \textsf{Prove that if \(H\) and \(K\) are normal subgroups of a group \(G\) then their intersection \(H \cap K\) is also a normal subgroup of \(G\).}
		\item \textsf{Prove that the intersection of an arbitrary non-empty collection of normal subgroups of a group is a normal subgroup (do not assume the collection is countable).}
	\end{enumerate}
\end{exercise}{}

\begin{proof}
	\begin{enumerate}[label=\emph{\alph*}]
		\item Denote $H \cap K = L$
		      \[
			      \forall l_1, l_2 \in L: l_1l_2 \in H, l_1l_2 \in K \implies l_1l_2 \in L
		      \]

		      \[
			      \forall l \in L: l^{-1} \in H, l^{-1} \in K \implies l^{-1} \in L
		      \]

		      \[
			      \forall l \in L: glg^{-1} \in H, glg^{-1} \in K \implies glg^{-1} \in L
		      \]
		\item There are no difference between the two argument, but just suggesting the element is in all normal subgroups instead of $H,K$.
	\end{enumerate}
\end{proof}

\begin{exercise}{24}
	\textsf{Prove that if \(N \trianglelefteq G\) and \(H\) is any subgroup of \(G\) then \(N \cap H \trianglelefteq H\).}
\end{exercise}{}

\begin{proof}
	By definition: $\forall g \in G: gN=Ng \implies \forall h \in H: hN=Nh$.
	Because $H$ is always normal under $H$.
	Then, follow last question, we can conclude that $N \cap H \trianglelefteq H$.
\end{proof}

\begin{exercise}{36}
	Prove that if \(G/Z(G)\) is cyclic then \(G\) is Abelian. [If \(G/Z(G)\) is cyclic with generator \(xZ(G)\), show that every element of \(G\) can be written in the form \(x^{a}z\) for some integer \(a \in \mathbb{Z}\) and some element \(z \in Z(G)\).]
\end{exercise}{}

\begin{proof}
	By definition, $Z(G) = \left\{ z \in G | zg = gz \right\}$.

	If ${G} / Z(G)$ is cyclic, we can write it with one generator $xZ(G)$, which means
	we can write every element as $x^aZ(G)^a=x^aZ(G)$.
	By definition of quotient group equivalent class,
	\[
		\forall g \in G: \exists a \in \Z, z \in Z(G) : g = x^az
	\]

	\[
		\forall g_1,g_2 \in G: g_1g_2 = x^{a_1}z_1x^{a_2}z_2 = z_1x^{a_1}x^{a_2}z_2=z_1x^{a_1+a_2}z_2=z_1z_2x^{a_1+a_2}=z_2z_1x^{a_2}x^{a_1}=z_2x^{a_2}z_1x^{a_1}=g_2g_1
	\]
\end{proof}

\begin{exercise}{40}
	\textsf{Let \(G\) be a group, let \(N\) be a normal subgroup of \(G\) and let \(\overline{G} = G/N\). Prove that \(\overline{x}\) and \(\overline{y}\) commute in \(\overline{G}\) if and only if \(x^{-1}y^{-1}xy \in N\). (The element \(x^{-1}y^{-1}xy\) is called the commutator of \(x\) and \(y\) and is denoted by \([x, y]\).)}
\end{exercise}{}

\begin{proof}
	$\impliedby$

	\[
		x^{-1}y^{-1}xy \in N \implies x \sim x x^{-1} y^{-1}xy = y^{-1}xy \implies yx \sim yy^{-1}xy = xy
	\]

	$\implies$

	If $\overline{x}\overline{y}=\overline{y}\overline{x}$

	\[
		\forall x,y \in \overline{x}\overline{y} : \overline{xy}=\overline{yx} \implies \overline{x^{-1}y^{-1}xy}=\one \implies x^{-1}y^{-1}xy \in N
	\]
\end{proof}

\begin{exercise}{41}
	\textsf{Let \(G\) be a group. Prove that \(N = \langle x^{-1}y^{-1} xy | x, y \in G \rangle\) is a normal subgroup of \(G\) and \(G / N\) is abelian (\(N\) is called the commutator subgroup of \(G\))}
\end{exercise}{}

\begin{proof}
	Closed under multiplication and inversion is trivial.

	Normal:

	Given Lemma, \(g^{-1}\langle S \rangle g = \langle g^{-1} S g \rangle\), it suffices to prove the \(S\) is normal.

	\[
		\forall g \in G: \forall x,y \in G: g(x^{-1}y^{-1}xy)g^{-1} = (gx^{-1}g^{-1})(gy^{-1}g^{-1})(gxg^{-1})(gyg^{-1})
	\]

	\[
		(gx^{-1}g^{-1})(gy^{-1}g^{-1})(gxg^{-1})(gyg^{-1}) = (gxg^{-1})^{-1}(gyg^{-1})^{-1}(gxg^{-1})(gyg^{-1}) \in N
	\]

	We can prove the lemma by following:

	Given \(g^{-1}Sg=S\), want to show \(g^{-1}S^ng\)

	\[
		g^{-1}S^ng = g^{-1}S^{n-1}Sg=g^{-1}S^{n-1}gg^{-1}Sg=g^{-1}S^{n-1}g
	\]

	By induction, we can see that it is equivalent.

	G/N is Abelian:

	This is obvious given question 40.
\end{proof}

\end{document}





