%: CLASS FILE

\documentclass[11pt]{article}
%\documentclass[11pt,reqno]{amsart} % AMS article style with right equation numbering


%: PACKAGES

\usepackage{graphicx}

% sets margins
\usepackage[top=1in,bottom=1in,left=1in,right=1in]{geometry}

% gives more color definitions
\usepackage{xcolor}

% gives optional arguments to list environments
\usepackage{enumitem}

\usepackage{microtype}

% uncomment to begin paragraphs with an empty line rather than an indent
% \usepackage[parfill]{parskip}

\usepackage{amsmath, amsthm, amssymb, tikz, mathtools, array, mathrsfs, tensor}

% must be loaded after ams packages
% gives various integral symbols: see http://ctan.math.illinois.edu/macros/latex/contrib/esint/esint.pdf
\usepackage{esint}

% creates command \one for blackboard 1 to use for indicator functions
\usepackage{dsfont}
	\newcommand{\one}{\mathds{1}}

% \framebox{$ stuff $} will frame things in line.
% \Aboxed{$ stuff $} will frame things in align environment.
\usepackage{framed}

% always have this as the last package listed.
\usepackage{hyperref}
	\hypersetup{hypertexnames=false,colorlinks=true,linkcolor=blue,citecolor=black,urlcolor=blue}
	
	
%: SETTINGS

% uncomment to align environments to be split over page breaks
%\allowdisplaybreaks

%: COMMANDS

% equation shortcuts
\newcommand{\eq}[1]{\begin{align*} #1 \end{align*}}
\newcommand{\eeq}[1]{\begin{align} \begin{split} #1 \end{split} \end{align}}

% shortcut for stacking equation references over relations
% usage: \stackref{LABEL}{SYMBOL}
\newcommand{\stackref}[2]{\stackrel{\mbox{\footnotesize{\eqref{#1}}}}{#2}}
\newcommand{\stackrefp}[2]{\stackrel{\phantom{\mbox{\footnotesize{\eqref{#1}}}}}{#2}}

% variable Greek letters
\def\eps{\varepsilon}
\def\vphi{\varphi}

% blackboard letters
\newcommand{\E}{\mathbb{E}}
\newcommand{\F}{\mathbb{F}}
\newcommand{\N}{\mathbb{N}}
\renewcommand{\P}{\mathbb{P}}
\newcommand{\Q}{\mathbb{Q}}
\newcommand{\R}{\mathbb{R}}
\renewcommand{\S}{\mathbb{S}}
\newcommand{\T}{\mathbb{T}}
\newcommand{\Z}{\mathbb{Z}}

% calligraphic letters
\renewcommand{\AA}{\mathcal{A}}
\newcommand{\BB}{\mathcal{B}}
\newcommand{\CC}{\mathcal{C}}
\newcommand{\DD}{\mathcal{D}}
\newcommand{\EE}{\mathcal{E}}
\newcommand{\FF}{\mathcal{F}}
\newcommand{\GG}{\mathcal{G}}
\newcommand{\HH}{\mathcal{H}}
\newcommand{\II}{\mathcal{I}}
\newcommand{\JJ}{\mathcal{J}}
\newcommand{\KK}{\mathcal{K}}
\newcommand{\LL}{\mathcal{L}}
\newcommand{\MM}{\mathcal{M}}
\newcommand{\NN}{\mathcal{N}}
\newcommand{\OO}{\mathcal{O}}
\newcommand{\PP}{\mathcal{P}}
\newcommand{\QQ}{\mathcal{Q}}
\newcommand{\RR}{\mathcal{R}}
\renewcommand{\SS}{\mathcal{S}}
\newcommand{\TT}{\mathcal{T}}
\newcommand{\UU}{\mathcal{U}}
\newcommand{\VV}{\mathcal{V}}
\newcommand{\WW}{\mathcal{W}}
\newcommand{\XX}{\mathcal{X}}
\newcommand{\YY}{\mathcal{Y}}
\newcommand{\ZZ}{\mathcal{Z}}

% script letters
\newcommand{\AAA}{\mathcal{A}}
\newcommand{\BBB}{\mathcal{B}}
\newcommand{\CCC}{\mathcal{C}}
\newcommand{\DDD}{\mathcal{D}}
\newcommand{\EEE}{\mathcal{E}}
\newcommand{\FFF}{\mathcal{F}}
\newcommand{\GGG}{\mathcal{G}}
\newcommand{\HHH}{\mathcal{H}}
\newcommand{\III}{\mathcal{I}}
\newcommand{\JJJ}{\mathcal{J}}
\newcommand{\KKK}{\mathcal{K}}
\newcommand{\LLL}{\mathcal{L}}
\newcommand{\MMM}{\mathcal{M}}
\newcommand{\NNN}{\mathcal{N}}
\newcommand{\OOO}{\mathcal{O}}
\newcommand{\PPP}{\mathcal{P}}
\newcommand{\QQQ}{\mathcal{Q}}
\newcommand{\RRR}{\mathcal{R}}
\newcommand{\SSS}{\mathcal{S}}
\newcommand{\TTT}{\mathcal{T}}
\newcommand{\UUU}{\mathcal{U}}
\newcommand{\VVV}{\mathcal{V}}
\newcommand{\WWW}{\mathcal{W}}
\newcommand{\XXX}{\mathcal{X}}
\newcommand{\YYY}{\mathcal{Y}}
\newcommand{\ZZZ}{\mathcal{Z}}



% inner product 
% usage: \iprod{left}{right}
\newcommand{\iprod}[2]{\langle #1,\, #2\rangle}

% bold math font (still in italics) 
% usage: {\vc CONTENT}
\newcommand{\vc}[1]{{\boldsymbol #1}}

% widetilde and widehat shortcuts
\newcommand{\wt}[1]{\widetilde{#1}}
\newcommand{\wh}[1]{\widehat{#1}}

% math operators with subcripts
\DeclareMathOperator{\Var}{Var}
\DeclareMathOperator{\Cov}{Cov}
\DeclareMathOperator{\Ent}{Ent}

% math operators with subtext
\DeclareMathOperator*{\esssup}{ess\,sup}
\DeclareMathOperator*{\essinf}{ess\,inf}

% conditional statements (size of delimiter as an optional command)
\newcommand{\givenc}[3][]{#1\{ #2 \: #1| \: #3 #1\}} % with braces
\newcommand{\givenk}[3][]{#1[ #2 \: #1| \: #3 #1]} % with brackets
\newcommand{\givenp}[3][]{#1( #2 \: #1| \: #3 #1)} % with parentheses
\newcommand{\givena}[3][]{#1\langle #2 \: #1| \: #3 #1\rangle} % with angle brackets

% common math text
\DeclareMathOperator{\e}{e} % exponential
\newcommand{\cc}{\mathrm{c}} % for set complements and "critical" subscripts
\newcommand{\dd}{\mathrm{d}} % for differentials
\newcommand{\Exp}{\mathrm{Exp}} % for exponential distribution

% floor and ceiling functions
\DeclarePairedDelimiter\ceil{\lceil}{\rceil}
\DeclarePairedDelimiter\floor{\lfloor}{\rfloor}

%: ENVIRONMENTS

%: And Or
\newcommand{\AND}{\text{ and }}
\newcommand{\OR}{\text{ or }}


% exercise environment
\newenvironment{exercise}[1]
	{\noindent \textbf{#1:}}
	{\par \vspace{0.5\baselineskip}}
	
% solution environment
\newenvironment{solution}[1][\unskip]
	{\noindent \textbf{Solution #1:} }
	{\qed \pagebreak}
	
\newenvironment{solutionn}[1][\unskip]
	{\noindent \textbf{Solution #1:} }
	{\qed \\}	


% theorem

\newtheorem{theorem}{Theorem}

%: TITLE

\begin{document}
\bibliographystyle{acm}

\begin{center}
	\framebox{\parbox{\linewidth}{\centering
	{\bf{Homework 5}}\\
	MATH 541: Abstract Algebra 1\\
	Spring 2023 \\[\baselineskip]
	{\sc Hongtao Zhang}}} %at \href{mailto:ewbates@wisc.edu}{\nolinkurl{ewbates@wisc.edu}}.}}
\end{center}


%: BODY

Section 3.3: 3, 4, 7, 9, 10

\section*{3}

\begin{exercise}{3}
	Prove that if $H$ is a normal subgroup of $G$ of prime index $p$ then for all $K \le G$ either
	\begin{enumerate}
		\item $K\leqslant H$ or
		\item $G=HK$ and $|K:K \cap H| = p$.
	\end{enumerate}
\end{exercise}

\begin{solution}
	Assume that $K \not\le H$

	By second isomorphism theorem,

	\[
		\frac{K}{K \cap H} \cong \frac{HK}{H}
		\implies |K:K \cap H| = |HK:H|
	\]

	If $K \le H$,
	\[
		|K:K \cap H| = |K:K| = 1 = |HK : H|
	\]

	Otherwise,
	\[
		\exists z > 1 \in Z^+ : |HK : H| = z
	\]

	\[
		|G : H| = p = {|G : HK|}{|HK : H|}
	\]

	Because $HK \le G \AND H \le HK$, both upper side and lower side is integer.

	We also know that $HK > H \impliedby K \not\le H$,
	so $|HK : H| \neq 1$,
	which $\implies |G:HK|=1 \implies G = HK$\dots

	Then $|K : K \cap H| = |HK : H| = p$.

\end{solution}

\section*{4}

\begin{exercise}{4}
	Let $C$	be a normal subgroup of the group $A$ and let $D$ be a normal subgroup of $B$.
	Prove that $(C \times D) \trianglelefteq (A \times B)$ and
	$(A \times B) / (C \times D) \cong (A / C) \times (B / D)$.
\end{exercise}

\begin{solution}
	Proof of finite cases.

	It is easy to see that $(C \times D) \trianglelefteq (A \times B)$,
	by definition of normal subgroup.

	\[
		\forall (a,b) \in (A \times B) : \forall  c_1,d_1 \in C,D, \exists c_2, d_2 \in C,D :
		(c_1a,d_1b) \in (C \times D) = (c_2a,d_2b) \in (C \times D)
	\]

	If $A,B,C,D$ is finite, the second statement follows directly from the Lagrange theorem.

	Proof of infinite cases.

	It follows from definition of how we construct quotient group.

	We can write element in $(A \times B)/(C \times D)$	as $(a,b)(C \times D)$.

	Then we have

	Denote element in $A$ as $a$ or $a_i$, and mutatis mutandis for $B,C,D$.

	Denote element in $A/C$ as $\overline{a}$ and mutatis mutandis for $B/D$, and $(A \times B) / (C \times D)$.

	Claim: map
	\[
		\phi(\overline{(a, b)}) = (\overline{a}, \overline{b})
	\]

	is a bijection.

	Proof of claim.

	It is a bijection because it is a function from a set to itself.

	\[
		\begin{split}
			& \forall (a,b) \in (A \times B) : \forall  c_1,d_1 \in C,D, \exists c_2, d_2 :
			(c_1a,d_1b) \in (C \times D) = (c_2a,d_2b)\\
			& \implies \forall (a,b) \in (A \times B) : \forall  c_1,d_1 \in C,D, \exists c_2, d_2 :
			(c_1, d_1)(a,b) = (c_2, d_2)(a,b)\\
		\end{split}
	\]



\end{solution}

\begin{exercise}{7}
	Let $M$ and $N$ be normal subgroups of $G$ such that $G=MN$.
	Prove that $G / (M \cap N) \cong (G / M) \times (G / N)$.
\end{exercise}

\begin{solution}
	It suffices to show that $(M \cap N)$ is the kernel of
	a morphism from $G \to (G / M) \times (G / N)$, by send
	$g \in G$ to $(gM, gN)$, which is clear.

	It is also clear that $M\cap N$ is the kernel of the morphism from $G$ to $G/(M \cap N)$.

	Because we know that both morphisms are surjective, we know that they are isomorphic.

\end{solution}

\begin{exercise}{9}
	Let $p$ be a prime and let $G$ be a group of order $p^am$,
	where $p$ does not divide $m$.
	Assume $P$ is a subgroup of $G$ of order $p^a$,
	and $N$ is a normal subgroup of $G$ of order $p^bn$,
	where $p$ does not divide $n$.
	Prove that $|P \cap N| = p^b$ and $|PN / N| = p^{a-b}$.
\end{exercise}

\begin{solution}
	By second isomorphism theorem,

	\[
		\frac{PN}{N} \cong \frac{P}{P \cap N}
		\implies \frac{(|PN|)}{(|N|)} = \frac{(|P|)}{(|P \cap N|)}
	\]

	We also know that $P \cap N \le N,P$

	Therefore,

	\[
		\begin{split}
			\exists z_1 \in Z^+ : = \frac{(|N|)}{|(P \cap N)|} \\
			\exists z_2 \in Z^+ : = \frac{(|P|)}{|(P \cap N)|} \\
		\end{split}
	\]

	Therefore,

	\[
		\begin{split}
			|N| = |P \cap N| \cdot z_1 \AND |P| = |P \cap N| \cdot z_2\\
			\implies \frac{|P|}{|P \cap N|} = \frac{|PN|}{z_1|P \cap N|}
			\implies z_1= |PN| / |P| = \frac{|PN|}{p^a} \\
		\end{split}
	\]

	We know that $PN \leqslant G$ so $|PN| \leqslant |G| \AND |PN| \backslash |G|$.

	Therefore,
	\[
		\exists x \in Z^+ : |PN| = p^ax \implies z_1 = x
	\]

	We also know that $N \leqslant PN \implies x \backslash n$.
	However, $|N| \backslash x \implies n \backslash x \implies n=x$.

	Therefore, $|PN| = p^an$.

	Then, by second isomorphism theorem,

	\[
	\frac{|PN|}{|N|} = \frac{|P|}{|P \cap N|} = p^{a-b} \implies |P \cap N| = p^b
	\]
\end{solution}

\begin{exercise}{10}
	Generalize the preceding exercise as follows.

	A subgroup $H$ of a finite group $G$ is called a \textit{Hall subgroup}
	of $G$ if its index in G is relatively prime to the order of $H$ 
	(i.e. $\operatorname{gcd}(|G:H|,|H|)=1$).
	Prove that if $H$ is a Hall subgroup of $G$ and $N$ is a normal subgroup of $G$,
	then $H \cap N$ is a Hall subgroup of $N$, and $HN/N$ is a Hall subgroup of $G / N$.
\end{exercise}

\begin{solution}
	In the previous question, we only use the fact
	that $p^a$ is relatively prime to $m \AND n$.

	Therefore, it suffices to write the order of $G$ as $xy$,
	where $x$ and $y$ are relatively prime, and $|H| = x$,
	and $|N| = an$ where $\operatorname{gcd}(a,n)=1$ and $x\backslash a, y\backslash n$.

	Then, we know that $|G:H| = y$.

	By previous question we know that $|H \cap N| = \frac{x}{a}$,
	and $|HN / N| = \frac{x}{a}$.

	Then it is clear that $H \cap N \leqslant N$ is a 
	Hall subgroup of $N$.

	Similarly, $|G / N| = \frac{x}{a} \frac{y}{n}$,
	which means that $HN / N$ is a Hall subgroup of $G / N$
	because $\frac{y}{n}$ will also be relatively prime to $\frac{x}{a}$.
\end{solution}

\end{document}





