%: CLASS FILE

\documentclass[11pt]{article}
%\documentclass[11pt,reqno]{amsart} % AMS article style with right equation numbering


%: PACKAGES

\usepackage{graphicx}

% sets margins
\usepackage[top=1in,bottom=1in,left=1in,right=1in]{geometry}

% gives more color definitions
\usepackage{xcolor}

% gives optional arguments to list environments
\usepackage{enumitem}

\usepackage{microtype}

% uncomment to begin paragraphs with an empty line rather than an indent
% \usepackage[parfill]{parskip}

\usepackage{amsmath, amsthm, amssymb, tikz, mathtools, array, mathrsfs, tensor}

% must be loaded after ams packages
% gives various integral symbols: see http://ctan.math.illinois.edu/macros/latex/contrib/esint/esint.pdf
\usepackage{esint}

% creates command \one for blackboard 1 to use for indicator functions
\usepackage{dsfont}
	\newcommand{\one}{\mathds{1}}

% \framebox{$ stuff $} will frame things in line.
% \Aboxed{$ stuff $} will frame things in align environment.
\usepackage{framed}

% always have this as the last package listed.
\usepackage{hyperref}
	\hypersetup{hypertexnames=false,colorlinks=true,linkcolor=blue,citecolor=black,urlcolor=blue}
	
	
%: SETTINGS

% uncomment to align environments to be split over page breaks
%\allowdisplaybreaks

%: COMMANDS

% equation shortcuts
\newcommand{\eq}[1]{\begin{align*} #1 \end{align*}}
\newcommand{\eeq}[1]{\begin{align} \begin{split} #1 \end{split} \end{align}}

% shortcut for stacking equation references over relations
% usage: \stackref{LABEL}{SYMBOL}
\newcommand{\stackref}[2]{\stackrel{\mbox{\footnotesize{\eqref{#1}}}}{#2}}
\newcommand{\stackrefp}[2]{\stackrel{\phantom{\mbox{\footnotesize{\eqref{#1}}}}}{#2}}

% variable Greek letters
\def\eps{\varepsilon}
\def\vphi{\varphi}

% blackboard letters
\newcommand{\E}{\mathbb{E}}
\newcommand{\F}{\mathbb{F}}
\newcommand{\N}{\mathbb{N}}
\renewcommand{\P}{\mathbb{P}}
\newcommand{\Q}{\mathbb{Q}}
\newcommand{\R}{\mathbb{R}}
\renewcommand{\S}{\mathbb{S}}
\newcommand{\T}{\mathbb{T}}
\newcommand{\Z}{\mathbb{Z}}

% calligraphic letters
\renewcommand{\AA}{\mathcal{A}}
\newcommand{\BB}{\mathcal{B}}
\newcommand{\CC}{\mathcal{C}}
\newcommand{\DD}{\mathcal{D}}
\newcommand{\EE}{\mathcal{E}}
\newcommand{\FF}{\mathcal{F}}
\newcommand{\GG}{\mathcal{G}}
\newcommand{\HH}{\mathcal{H}}
\newcommand{\II}{\mathcal{I}}
\newcommand{\JJ}{\mathcal{J}}
\newcommand{\KK}{\mathcal{K}}
\newcommand{\LL}{\mathcal{L}}
\newcommand{\MM}{\mathcal{M}}
\newcommand{\NN}{\mathcal{N}}
\newcommand{\OO}{\mathcal{O}}
\newcommand{\PP}{\mathcal{P}}
\newcommand{\QQ}{\mathcal{Q}}
\newcommand{\RR}{\mathcal{R}}
\renewcommand{\SS}{\mathcal{S}}
\newcommand{\TT}{\mathcal{T}}
\newcommand{\UU}{\mathcal{U}}
\newcommand{\VV}{\mathcal{V}}
\newcommand{\WW}{\mathcal{W}}
\newcommand{\XX}{\mathcal{X}}
\newcommand{\YY}{\mathcal{Y}}
\newcommand{\ZZ}{\mathcal{Z}}

% script letters
\newcommand{\AAA}{\mathcal{A}}
\newcommand{\BBB}{\mathcal{B}}
\newcommand{\CCC}{\mathcal{C}}
\newcommand{\DDD}{\mathcal{D}}
\newcommand{\EEE}{\mathcal{E}}
\newcommand{\FFF}{\mathcal{F}}
\newcommand{\GGG}{\mathcal{G}}
\newcommand{\HHH}{\mathcal{H}}
\newcommand{\III}{\mathcal{I}}
\newcommand{\JJJ}{\mathcal{J}}
\newcommand{\KKK}{\mathcal{K}}
\newcommand{\LLL}{\mathcal{L}}
\newcommand{\MMM}{\mathcal{M}}
\newcommand{\NNN}{\mathcal{N}}
\newcommand{\OOO}{\mathcal{O}}
\newcommand{\PPP}{\mathcal{P}}
\newcommand{\QQQ}{\mathcal{Q}}
\newcommand{\RRR}{\mathcal{R}}
\newcommand{\SSS}{\mathcal{S}}
\newcommand{\TTT}{\mathcal{T}}
\newcommand{\UUU}{\mathcal{U}}
\newcommand{\VVV}{\mathcal{V}}
\newcommand{\WWW}{\mathcal{W}}
\newcommand{\XXX}{\mathcal{X}}
\newcommand{\YYY}{\mathcal{Y}}
\newcommand{\ZZZ}{\mathcal{Z}}

% inner product 
% usage: \iprod{left}{right}
\newcommand{\iprod}[2]{\langle #1,\, #2\rangle}

% bold math font (still in italics) 
% usage: {\vc CONTENT}
\newcommand{\vc}[1]{{\boldsymbol #1}}

% widetilde and widehat shortcuts
\newcommand{\wt}[1]{\widetilde{#1}}
\newcommand{\wh}[1]{\widehat{#1}}

% math operators with subcripts
\DeclareMathOperator{\Var}{Var}
\DeclareMathOperator{\Cov}{Cov}
\DeclareMathOperator{\Ent}{Ent}

% math operators with subtext
\DeclareMathOperator*{\esssup}{ess\,sup}
\DeclareMathOperator*{\essinf}{ess\,inf}

% conditional statements (size of delimiter as an optional command)
\newcommand{\givenc}[3][]{#1\{ #2 \: #1| \: #3 #1\}} % with braces
\newcommand{\givenk}[3][]{#1[ #2 \: #1| \: #3 #1]} % with brackets
\newcommand{\givenp}[3][]{#1( #2 \: #1| \: #3 #1)} % with parentheses
\newcommand{\givena}[3][]{#1\langle #2 \: #1| \: #3 #1\rangle} % with angle brackets

% common math text
\DeclareMathOperator{\e}{e} % exponential
\newcommand{\cc}{\mathrm{c}} % for set complements and "critical" subscripts
\newcommand{\dd}{\mathrm{d}} % for differentials
\newcommand{\Exp}{\mathrm{Exp}} % for exponential distribution

% floor and ceiling functions
\DeclarePairedDelimiter\ceil{\lceil}{\rceil}
\DeclarePairedDelimiter\floor{\lfloor}{\rfloor}

%: ENVIRONMENTS

% exercise environment
\newenvironment{exercise}[1]
	{\noindent \textbf{#1:}}
	{\par \vspace{0.5\baselineskip}}
	
% solution environment
\newenvironment{solution}[1][\unskip]
	{\noindent \textbf{Solution #1:} }
	{\qed \pagebreak}
	
\newenvironment{solutionn}[1][\unskip]
	{\noindent \textbf{Solution #1:} }
	{\qed \\}	


\newtheorem{lemma}{Lemma}

%: TITLE

\begin{document}
\bibliographystyle{acm}

\begin{center}
	\framebox{\parbox{\linewidth}{\centering
	{\bf{Honors 1}}\\
	MATH 541: Abstract Algebra 1\\
	Spring 2023 \\[\baselineskip]
	{\sc Hongtao Zhang, Yibo Wu}}} %at \href{mailto:ewbates@wisc.edu}{\nolinkurl{ewbates@wisc.edu}}.}}
\end{center}


%: BODY

\begin{lemma}\label{lem1}
	a,b,c,d are coprime.
\end{lemma}

\begin{proof}
	Assume they are not coprime,
	without loss of generality, let $\operatorname{gcd}(a,c)=k$

	\[
		(ad-bc)=k(ld-nb) =1 \implies k=1
	\]
\end{proof}

We can tackle this problem by utilizing the algorithm of finding a matrix inverse.
To find a matrix inverse, we can use the following algorithm:

\begin{enumerate}
	\item Do row operation to make the matrix into an upper triangular matrix.
	\item Do row operation to make the matrix into an identity matrix.
	\item The inverse of the original matrix is the matrix we get from the elementary row operation matrix product.
\end{enumerate}

We know that $I$ can be written as $A^{0}$,
if we can represent all the row operation needed to reduce $S \in SL_2(\Z)$ to $I$,
then we can represent $S^{-1}$ as $A^{0}A^{1}A^{2}...A^{n}$,
where $A^{i}$ is the elementary row operation matrix,
which means we can reproduce $S$.

\begin{lemma}
	Row operation for adding $k \in \Z$ times the bottom row to the top row can be represented as $A^{k}$.
\end{lemma}

\begin{proof}
	Proof is trivial so left as an exercise to the reader.
\end{proof}

\begin{lemma}
	Row operation for adding $k \in \Z$ times the top row to the bottom row by $k$ can be represented as $B^{-3}A^{-k}B$.
\end{lemma}

\begin{proof}
	\begin{align*}
		B^{-3}A^{-k}B & = \begin{bmatrix}
			                  0 & -1 \\
			                  1 & 0  \\
		                  \end{bmatrix}^{-3}  \begin{bmatrix}
			                                      1 & 1 \\
			                                      0 & 1 \\
		                                      \end{bmatrix}^{-k}\begin{bmatrix}
			                                                        0 & -1 \\
			                                                        1 & 0  \\
		                                                        \end{bmatrix} \\
	\end{align*}

	through matrix multiplication

	\begin{align*}
		B^{-3}A^{-k}B & = \begin{bmatrix}
			                  1 & 0 \\
			                  k & 1
		                  \end{bmatrix} \\
	\end{align*}

	which means

	\begin{align*}
		B^{-3}A^{-k}B\begin{bmatrix}
			             a & b \\
			             c & d
		             \end{bmatrix} & = \begin{bmatrix}
			                               a    & b    \\
			                               ak+c & bk+d
		                               \end{bmatrix} \\
	\end{align*}
\end{proof}

Then what we need to show is we can reduce to $I$ without scaling multiplication of a row.

By lemma \ref{lem1}, we know that $a,b,c,d$ are coprime.

Therefore, $\operatorname{gcd}(a,c)=1$.

If we do row operation following the Euclidean algorithm,
we are guaranteed to reduce $a,c$ to be $1$.
Then, if we do one more time, we can make $c$ to be $0$.

We know that $ad-bc=1$, with $c=0, a=1$, we can get $d=1$.

Therefore, by subtracting the bottom row from the top row the remaining $b$ times,
we can get $I$.

Therefore, we can reduce $S$ to $I$, which means $A,B$ can 
formulate $S^{-1}$ by multiplication and inversion, 
which by one more inverse, we can get $S$.

\qed


\end{document}





