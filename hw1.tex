%: CLASS FILE

\documentclass[11pt]{article}
%\documentclass[11pt,reqno]{amsart} % AMS article style with right equation numbering


%: PACKAGES

\usepackage{graphicx}

% sets margins
\usepackage[top=1in,bottom=1in,left=1in,right=1in]{geometry}

% gives more color definitions
\usepackage{xcolor}

% gives optional arguments to list environments
\usepackage{enumitem}

\usepackage{microtype}

% uncomment to begin paragraphs with an empty line rather than an indent
% \usepackage[parfill]{parskip}

\usepackage{amsmath, amsthm, amssymb, tikz, mathtools, array, mathrsfs, tensor}

% must be loaded after ams packages
% gives various integral symbols: see http://ctan.math.illinois.edu/macros/latex/contrib/esint/esint.pdf
\usepackage{esint}

% creates command \one for blackboard 1 to use for indicator functions
\usepackage{dsfont}
	\newcommand{\one}{\mathds{1}}

% \framebox{$ stuff $} will frame things in line.
% \Aboxed{$ stuff $} will frame things in align environment.
\usepackage{framed}

% always have this as the last package listed.
\usepackage{hyperref}
	\hypersetup{hypertexnames=false,colorlinks=true,linkcolor=blue,citecolor=black,urlcolor=blue}
	
	
%: SETTINGS

% uncomment to align environments to be split over page breaks
%\allowdisplaybreaks

%: COMMANDS

% equation shortcuts
\newcommand{\eq}[1]{\begin{align*} #1 \end{align*}}
\newcommand{\eeq}[1]{\begin{align} \begin{split} #1 \end{split} \end{align}}

% shortcut for stacking equation references over relations
% usage: \stackref{LABEL}{SYMBOL}
\newcommand{\stackref}[2]{\stackrel{\mbox{\footnotesize{\eqref{#1}}}}{#2}}
\newcommand{\stackrefp}[2]{\stackrel{\phantom{\mbox{\footnotesize{\eqref{#1}}}}}{#2}}

% variable Greek letters
\def\eps{\varepsilon}
\def\vphi{\varphi}

% blackboard letters
\newcommand{\E}{\mathbb{E}}
\newcommand{\F}{\mathbb{F}}
\newcommand{\N}{\mathbb{N}}
\renewcommand{\P}{\mathbb{P}}
\newcommand{\Q}{\mathbb{Q}}
\newcommand{\R}{\mathbb{R}}
\renewcommand{\S}{\mathbb{S}}
\newcommand{\T}{\mathbb{T}}
\newcommand{\Z}{\mathbb{Z}}

% calligraphic letters
\renewcommand{\AA}{\mathcal{A}}
\newcommand{\BB}{\mathcal{B}}
\newcommand{\CC}{\mathcal{C}}
\newcommand{\DD}{\mathcal{D}}
\newcommand{\EE}{\mathcal{E}}
\newcommand{\FF}{\mathcal{F}}
\newcommand{\GG}{\mathcal{G}}
\newcommand{\HH}{\mathcal{H}}
\newcommand{\II}{\mathcal{I}}
\newcommand{\JJ}{\mathcal{J}}
\newcommand{\KK}{\mathcal{K}}
\newcommand{\LL}{\mathcal{L}}
\newcommand{\MM}{\mathcal{M}}
\newcommand{\NN}{\mathcal{N}}
\newcommand{\OO}{\mathcal{O}}
\newcommand{\PP}{\mathcal{P}}
\newcommand{\QQ}{\mathcal{Q}}
\newcommand{\RR}{\mathcal{R}}
\renewcommand{\SS}{\mathcal{S}}
\newcommand{\TT}{\mathcal{T}}
\newcommand{\UU}{\mathcal{U}}
\newcommand{\VV}{\mathcal{V}}
\newcommand{\WW}{\mathcal{W}}
\newcommand{\XX}{\mathcal{X}}
\newcommand{\YY}{\mathcal{Y}}
\newcommand{\ZZ}{\mathcal{Z}}

% script letters
\newcommand{\AAA}{\mathcal{A}}
\newcommand{\BBB}{\mathcal{B}}
\newcommand{\CCC}{\mathcal{C}}
\newcommand{\DDD}{\mathcal{D}}
\newcommand{\EEE}{\mathcal{E}}
\newcommand{\FFF}{\mathcal{F}}
\newcommand{\GGG}{\mathcal{G}}
\newcommand{\HHH}{\mathcal{H}}
\newcommand{\III}{\mathcal{I}}
\newcommand{\JJJ}{\mathcal{J}}
\newcommand{\KKK}{\mathcal{K}}
\newcommand{\LLL}{\mathcal{L}}
\newcommand{\MMM}{\mathcal{M}}
\newcommand{\NNN}{\mathcal{N}}
\newcommand{\OOO}{\mathcal{O}}
\newcommand{\PPP}{\mathcal{P}}
\newcommand{\QQQ}{\mathcal{Q}}
\newcommand{\RRR}{\mathcal{R}}
\newcommand{\SSS}{\mathcal{S}}
\newcommand{\TTT}{\mathcal{T}}
\newcommand{\UUU}{\mathcal{U}}
\newcommand{\VVV}{\mathcal{V}}
\newcommand{\WWW}{\mathcal{W}}
\newcommand{\XXX}{\mathcal{X}}
\newcommand{\YYY}{\mathcal{Y}}
\newcommand{\ZZZ}{\mathcal{Z}}

% inner product 
% usage: \iprod{left}{right}
\newcommand{\iprod}[2]{\langle #1,\, #2\rangle}

% bold math font (still in italics) 
% usage: {\vc CONTENT}
\newcommand{\vc}[1]{{\boldsymbol #1}}

% widetilde and widehat shortcuts
\newcommand{\wt}[1]{\widetilde{#1}}
\newcommand{\wh}[1]{\widehat{#1}}

% math operators with subcripts
\DeclareMathOperator{\Var}{Var}
\DeclareMathOperator{\Cov}{Cov}
\DeclareMathOperator{\Ent}{Ent}

% math operators with subtext
\DeclareMathOperator*{\esssup}{ess\,sup}
\DeclareMathOperator*{\essinf}{ess\,inf}

% conditional statements (size of delimiter as an optional command)
\newcommand{\givenc}[3][]{#1\{ #2 \: #1| \: #3 #1\}} % with braces
\newcommand{\givenk}[3][]{#1[ #2 \: #1| \: #3 #1]} % with brackets
\newcommand{\givenp}[3][]{#1( #2 \: #1| \: #3 #1)} % with parentheses
\newcommand{\givena}[3][]{#1\langle #2 \: #1| \: #3 #1\rangle} % with angle brackets

% common math text
\DeclareMathOperator{\e}{e} % exponential
\newcommand{\cc}{\mathrm{c}} % for set complements and "critical" subscripts
\newcommand{\dd}{\mathrm{d}} % for differentials
\newcommand{\Exp}{\mathrm{Exp}} % for exponential distribution

% floor and ceiling functions
\DeclarePairedDelimiter\ceil{\lceil}{\rceil}
\DeclarePairedDelimiter\floor{\lfloor}{\rfloor}

%: ENVIRONMENTS

% exercise environment
\newenvironment{exercise}[1]
	{\noindent \textbf{#1:}}
	{\par \vspace{0.5\baselineskip}}
	
% solution environment
\newenvironment{solution}[1][\unskip]
	{\noindent \textbf{Solution #1:} }
	{\qed \pagebreak}
	
\newenvironment{solutionn}[1][\unskip]
	{\noindent \textbf{Solution #1:} }
	{\qed \\}	


%: TITLE

\begin{document}
\bibliographystyle{acm}

\begin{center}
	\framebox{\parbox{\linewidth}{\centering
	{\bf{Homework 1}}\\
	MATH 541: Abstract Algebra 1\\
	Spring 2023 \\[\baselineskip]
	{\sc Hongtao Zhang}}} %at \href{mailto:ewbates@wisc.edu}{\nolinkurl{ewbates@wisc.edu}}.}}
\end{center}


%: BODY

Section 1.1: 1, 7

Section 1.2: 3, 4, 18

Section 1.3: 1, 5

\section*{1.1.1}

Clearly
$$
\begin{pmatrix} 
	0 & 0\\
	0 & 0 
\end{pmatrix},
\begin{pmatrix}
	1 & 0\\
	0 & 1
\end{pmatrix},
\begin{pmatrix}
	1 & 1\\
	1 & 1 
\end{pmatrix}  
$$
are satisfying the condtion.

Also, $M$ itself must also satisfy the condition.

$$
\begin{pmatrix}
	1 & 1 \\
	0 & 1 \\
\end{pmatrix}
\begin{pmatrix}
	1 & 1 \\
	1 & 0 \\
\end{pmatrix} = 
\begin{pmatrix}
	2 & 1 \\
	1 & 0 \\
\end{pmatrix}
$$

$$
\begin{pmatrix}
	1 & 1 \\
	1 & 0 \\
\end{pmatrix}
\begin{pmatrix}
	1 & 1 \\
	0 & 1 \\
\end{pmatrix} = 
\begin{pmatrix}
	1 & 2 \\
	1 & 1 \\
\end{pmatrix}
$$

Therefore it is not in B.

$$
\begin{pmatrix}
	0 & 1 \\
	1 & 0 \\
\end{pmatrix}M=
\begin{pmatrix}
	1 & 1 \\in
	1 & 0 \\
\end{pmatrix}
$$

$$
M\begin{pmatrix}
	0 & 1 \\
	1 & 0 \\
\end{pmatrix}=M
$$

Therefore, it is not in B.

\section*{1.1.7}

\begin{proof}
	To be an equivalence relation, we need three condtions.

	$$
	\begin{cases}
		a \sim a \\ 
		a \sim b \iff b \sim a\\
		a \sim b \land b \sim c \implies a \sim c
	\end{cases}
	$$

	It is very clear that $a \sim a$ because by the definitions of a function $f$, $f(a)=f(a)$.

	If $f(a)=f(b)$ then $f(b)=f(a)$ so $a \sim b \implies b \sim a$.

	$f(a)=f(b)\land f(b)=f(c)\implies f(a)=f(c)$, so $a \sim b \land b \sim c \implies a \sim c$.

	We know that the fibers of element $y$ are $\left\{ x \in X : f(x) = y \right\}$.
	Therefore, it is very clear that if $a \sim b$, then a,b are in the fibers of $f(a)$.
\end{proof}

\section*{1.2.3}

\begin{proof}
	$s^k$ has order 2 because $s^{2k}=(s^2)^k=1^k=1$.
	$sr^ksr^k=sr^{k-1}sr^{-1}r^k=sr^{k-1}sr^{k-1}$ so we can reduce to the case $s^2$ which is $1$.

	To prove that $s,sr$ can generate $D_{2n}$, we need to show that $r$ ($s$ is trivial) can be written as combination of $s$ and $sr$.
	This is trivial by compose $s$ and $sr$ which results in $r$.
\end{proof}

\section*{1.2.4}

\begin{proof}
	If $n=2k$ is even, $k$ is an integer. 

	For all element that are generated as power of $r$, it is very clear that it is commute with $r^k$\dots

	It is very clear that $(r^k)^2=r^n=1$. Therefore, $r^kr^{-k}=1=r^k \implies r^k=r^{-k}$.

	$zs=r^ks=r^{k-1}sr^{-1}=sr^{-k}=sr^k=sz$.

	Suppose an element $l=r^is^j$ that commute with all element in $D_{2n}$.

	It suffics to check two thing $(rl=lr, sl=ls)$.

	$$
	lr=rl \implies r^is^jr = rr^is^j \implies s^jr = rs^j
	$$

	$$
	\begin{cases}
		j \mod 2 = 0 \implies r = r \\ 
		j \mod 2 = 1 \implies sr = rs = sr^{-1} \implies r = r^{-1} (\text{contradiction})
	\end{cases}
	$$

	Therefore $j=0$.

	$$
	ls=sl \implies r^i s^j s = s r^i s^j \implies r^i s = s r^j \implies sr^{-i} = sr^{i} \implies r^{-i} = r^{i} \implies r^i r^i = r^i r^{-i} = 1
	$$

	Therefore $i=\frac{n}{2}$ since we assume $n$ is even.

\end{proof}

\section*{1.2.18}

\begin{enumerate}
	\item $v^3=1 \implies v^2 v = 1 \implies v^2 v = v^{-1} v \implies v^2 = v^{-1}$
	\item $(v^2 u^2) (uv) = (v^2 u^2) (v^2 u^2) = (uv) (v^2 u^2) = u v^3 u^2 = u^3$
	
		$v^2u^3v = u^3 = v^{-1} u^3 v = u^3 \implies u^3 v = v u^3$
	\item $u^4 = 1 \implies u^8 = 1 \implies u^8 u = u$. From before, $vu = v u^3 u^3 u^3 = u^3 v u^3 u^3 = u^3 u^3 u^3 v = u^9 v = uv$
	\item $vu = uv = v^2u^2 \implies u = vu^2 \implies 1 = vu = uv$
	\item $u^4v^3 = 1 = uv uv uv u = u \implies 1v=1 \implies v=1$
\end{enumerate}

\section*{1.3.1}

$$
\sigma = (1\ 3\ 5) (2\ 4)
$$

$$
\tau = (1\ 5)(2\ 3)
$$

$$
\sigma^2 = (1 5 3)
$$

$$
\sigma \tau = (1)(2\ 5\ 3\ 4)
$$

$$
\tau \sigma = (1\ 2\ 4\ 3) (5)
$$

$$
\tau^2 \sigma = \sigma = (1\ 3\ 5) (2\ 4)
$$

\section*{1.3.5}

\begin{proof}
	The order should be the gcd of the count of elements in each cycle group.

	$$
	\operatorname{gcd}(5,2,3,2) = 30
	$$

\end{proof}


\end{document}





