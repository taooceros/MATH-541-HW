%: CLASS FILE

\documentclass[11pt]{article}
%\documentclass[11pt,reqno]{amsart} % AMS article style with right equation numbering


%: PACKAGES

\usepackage{graphicx}

% sets margins
\usepackage[top=1in,bottom=1in,left=1in,right=1in]{geometry}

% gives more color definitions
\usepackage{xcolor}

% gives optional arguments to list environments
\usepackage{enumitem}

\usepackage{microtype}

% uncomment to begin paragraphs with an empty line rather than an indent
% \usepackage[parfill]{parskip}

\usepackage{amsmath, amsthm, amssymb, tikz, mathtools, array, mathrsfs, tensor}

% must be loaded after ams packages
% gives various integral symbols: see http://ctan.math.illinois.edu/macros/latex/contrib/esint/esint.pdf
\usepackage{esint}

% creates command \one for blackboard 1 to use for indicator functions
\usepackage{dsfont}
	\newcommand{\one}{\mathds{1}}

% \framebox{$ stuff $} will frame things in line.
% \Aboxed{$ stuff $} will frame things in align environment.
\usepackage{framed}

% always have this as the last package listed.
\usepackage{hyperref}
	\hypersetup{hypertexnames=false,colorlinks=true,linkcolor=blue,citecolor=black,urlcolor=blue}
	
	
%: SETTINGS

% uncomment to align environments to be split over page breaks
%\allowdisplaybreaks

%: COMMANDS

% equation shortcuts
\newcommand{\eq}[1]{\begin{align*} #1 \end{align*}}
\newcommand{\eeq}[1]{\begin{align} \begin{split} #1 \end{split} \end{align}}

% shortcut for stacking equation references over relations
% usage: \stackref{LABEL}{SYMBOL}
\newcommand{\stackref}[2]{\stackrel{\mbox{\footnotesize{\eqref{#1}}}}{#2}}
\newcommand{\stackrefp}[2]{\stackrel{\phantom{\mbox{\footnotesize{\eqref{#1}}}}}{#2}}

% variable Greek letters
\def\eps{\varepsilon}
\def\vphi{\varphi}

% blackboard letters
\newcommand{\E}{\mathbb{E}}
\newcommand{\F}{\mathbb{F}}
\newcommand{\N}{\mathbb{N}}
\renewcommand{\P}{\mathbb{P}}
\newcommand{\Q}{\mathbb{Q}}
\newcommand{\R}{\mathbb{R}}
\renewcommand{\S}{\mathbb{S}}
\newcommand{\T}{\mathbb{T}}
\newcommand{\Z}{\mathbb{Z}}

% calligraphic letters
\renewcommand{\AA}{\mathcal{A}}
\newcommand{\BB}{\mathcal{B}}
\newcommand{\CC}{\mathcal{C}}
\newcommand{\DD}{\mathcal{D}}
\newcommand{\EE}{\mathcal{E}}
\newcommand{\FF}{\mathcal{F}}
\newcommand{\GG}{\mathcal{G}}
\newcommand{\HH}{\mathcal{H}}
\newcommand{\II}{\mathcal{I}}
\newcommand{\JJ}{\mathcal{J}}
\newcommand{\KK}{\mathcal{K}}
\newcommand{\LL}{\mathcal{L}}
\newcommand{\MM}{\mathcal{M}}
\newcommand{\NN}{\mathcal{N}}
\newcommand{\OO}{\mathcal{O}}
\newcommand{\PP}{\mathcal{P}}
\newcommand{\QQ}{\mathcal{Q}}
\newcommand{\RR}{\mathcal{R}}
\renewcommand{\SS}{\mathcal{S}}
\newcommand{\TT}{\mathcal{T}}
\newcommand{\UU}{\mathcal{U}}
\newcommand{\VV}{\mathcal{V}}
\newcommand{\WW}{\mathcal{W}}
\newcommand{\XX}{\mathcal{X}}
\newcommand{\YY}{\mathcal{Y}}
\newcommand{\ZZ}{\mathcal{Z}}

% script letters
\newcommand{\AAA}{\mathcal{A}}
\newcommand{\BBB}{\mathcal{B}}
\newcommand{\CCC}{\mathcal{C}}
\newcommand{\DDD}{\mathcal{D}}
\newcommand{\EEE}{\mathcal{E}}
\newcommand{\FFF}{\mathcal{F}}
\newcommand{\GGG}{\mathcal{G}}
\newcommand{\HHH}{\mathcal{H}}
\newcommand{\III}{\mathcal{I}}
\newcommand{\JJJ}{\mathcal{J}}
\newcommand{\KKK}{\mathcal{K}}
\newcommand{\LLL}{\mathcal{L}}
\newcommand{\MMM}{\mathcal{M}}
\newcommand{\NNN}{\mathcal{N}}
\newcommand{\OOO}{\mathcal{O}}
\newcommand{\PPP}{\mathcal{P}}
\newcommand{\QQQ}{\mathcal{Q}}
\newcommand{\RRR}{\mathcal{R}}
\newcommand{\SSS}{\mathcal{S}}
\newcommand{\TTT}{\mathcal{T}}
\newcommand{\UUU}{\mathcal{U}}
\newcommand{\VVV}{\mathcal{V}}
\newcommand{\WWW}{\mathcal{W}}
\newcommand{\XXX}{\mathcal{X}}
\newcommand{\YYY}{\mathcal{Y}}
\newcommand{\ZZZ}{\mathcal{Z}}

% inner product 
% usage: \iprod{left}{right}
\newcommand{\iprod}[2]{\langle #1,\, #2\rangle}

% bold math font (still in italics) 
% usage: {\vc CONTENT}
\newcommand{\vc}[1]{{\boldsymbol #1}}

% widetilde and widehat shortcuts
\newcommand{\wt}[1]{\widetilde{#1}}
\newcommand{\wh}[1]{\widehat{#1}}

% math operators with subcripts
\DeclareMathOperator{\Var}{Var}
\DeclareMathOperator{\Cov}{Cov}
\DeclareMathOperator{\Ent}{Ent}

% math operators with subtext
\DeclareMathOperator*{\esssup}{ess\,sup}
\DeclareMathOperator*{\essinf}{ess\,inf}

% conditional statements (size of delimiter as an optional command)
\newcommand{\givenc}[3][]{#1\{ #2 \: #1| \: #3 #1\}} % with braces
\newcommand{\givenk}[3][]{#1[ #2 \: #1| \: #3 #1]} % with brackets
\newcommand{\givenp}[3][]{#1( #2 \: #1| \: #3 #1)} % with parentheses
\newcommand{\givena}[3][]{#1\langle #2 \: #1| \: #3 #1\rangle} % with angle brackets

% common math text
\DeclareMathOperator{\e}{e} % exponential
\newcommand{\cc}{\mathrm{c}} % for set complements and "critical" subscripts
\newcommand{\dd}{\mathrm{d}} % for differentials
\newcommand{\Exp}{\mathrm{Exp}} % for exponential distribution

% floor and ceiling functions
\DeclarePairedDelimiter\ceil{\lceil}{\rceil}
\DeclarePairedDelimiter\floor{\lfloor}{\rfloor}

%: ENVIRONMENTS

% exercise environment
\newenvironment{exercise}[1]
	{\noindent \textbf{#1:}}
	{\par \vspace{0.5\baselineskip}}
	
% solution environment
\newenvironment{solution}[1][\unskip]
	{\noindent \textbf{Solution #1:} }
	{\qed \pagebreak}
	
\newenvironment{solutionn}[1][\unskip]
	{\noindent \textbf{Solution #1:} }
	{\qed \\}	


%: TITLE

\begin{document}
\bibliographystyle{acm}

\begin{center}
	\framebox{\parbox{\linewidth}{\centering
	{\bf{Homework 3}}\\
	MATH 541: Abstract Algebra 1\\
	Spring 2023 \\[\baselineskip]
	{\sc Hongtao Zhang}}} %at \href{mailto:ewbates@wisc.edu}{\nolinkurl{ewbates@wisc.edu}}.}}
\end{center}


%: BODY

Sec. 0.1: 7

Sec. 0.2: 7

Sec. 0.3: 8

Sec. 1.7: 18, 19, 23

Sec. 2.1: 3


\section*{0.1.7}

\begin{proof}
	To be an equivalence relation, we need three condtions.

	$$
		\begin{cases}
			a \sim a               \\
			a \sim b \iff b \sim a \\
			a \sim b \land b \sim c \implies a \sim c
		\end{cases}
	$$

	It is very clear that $a \sim a$ because by the definitions of a function $f$, $f(a)=f(a)$.

	If $f(a)=f(b)$ then $f(b)=f(a)$ so $a \sim b \implies b \sim a$.

	$f(a)=f(b)\land f(b)=f(c)\implies f(a)=f(c)$, so $a \sim b \land b \sim c \implies a \sim c$.

	We know that the fibers of element $y$ are $\left\{ x \in X : f(x) = y \right\}$.
	Therefore, it is very clear that if $a \sim b$, then a,b are in the fibers of $f(a)$.
\end{proof}

\section*{0.2.7}

\begin{proof}
	Proof by contradiction

	Assume there exists an $a$ such that $a^2=pb^2$,
	and write $a=\prod_{n}^{} p_{an}, b=\prod_{n}^{} p_{bn}  $
	then we know that $a^2=\left( \prod_{n}^{} p_{an} \right)^2 = pb^2 = p(\prod_{n}^{} p_{bn})^2$

	Because we know that for every integer, there's an unique prime decomposition, so the power of
	left primes must match the power of right primes.

	However, because we know that $p$ is a prime, and all primes component from $b$ will have even power,
	so the power of $p$ must be odd, which mismatched the power of $a$'s decomposition, which is a contradiction.
\end{proof}

\section*{0.3.8}

\subsection*{0.3.6}

\begin{proof}
	the square of $\bar{0}^2=\bar{0}$
	Assume we have an element $a$ in $\bar{1}$, write $a=(4b+1)$ for some integer $b$.
	$$
		a^2 = aa = (4b+1) (4b + 1) = 16b + 4b + 4b + 1 \mod 4 = 1
	$$

	For $\bar{2}$
	$$
		(4b+2)(4b+2) = 16b+8b+8b+4 \mod 4 = 0
	$$

	For $\bar{3}$
	$$
		(4b+3)(4b+3) \mod 4 = 9 \mod 4 = 1
	$$
\end{proof}

\subsection*{0.3.7}

\begin{proof}
	We know that $a^2,b^2 \mod 4 = 0 \text{ or } 1$, therefore $a^2+b^2 \mod 4 \le 2$
\end{proof}

\subsection*{0.3.8}

\begin{proof}
	Proof by contradiction

	Assume the solution exists.

	We know that $a^2+b^2 \mod 4 \neq 3$, so $c^2 \mod 4 \neq 1$, which means $c^2 \mod 4=0$

	Also we know that $a^2\mod 4 < 2$, and we know that $c^2 \mod 4 = 0$.

	Therefore, $(a^2 + b^2) \mod 4 = 0$

	Therefore $a^2 \mod 4 = b^2 \mod 4 = 0$.

	Therefore, $\frac{a^2}{4}, \frac{b^2}{4}, \frac{c^2}{4} \in \Z$.

	Therefore, $\frac{a^2}{4}, \frac{b^2}{4}, \frac{c^2}{4}$ satisfy the same constraint,
	so we can contiuously divide out by 4, and the equation still satisfy.

	However, it is impossible for $a^2, b^2, c^2$ to have infinite many factor of $4$, which is a contradiction.

\end{proof}

\section*{1.7.18}

\begin{proof}
	\begin{enumerate}
		\item Reflexivity. This is true because $\one \in H$, and $\one a = a$
		\item Symmetry. This is true because inverse.
		      $$
			      a \sim b \implies \exists h : ha=b \implies h^{-1} b = a
		      $$

		      The argument is symmetric so the other side is the same.

		\item transitivity. This is true because group is closed.
		      $$
			      a \sim b, b \sim c \implies \exists h_1,h_2 : h_1a=b, h_2b=c
		      $$
		      $$
			      \implies h_2 h_1 a = c \implies \exists k = h_2 h_1 \in H : ka=c \implies a \sim c
		      $$
	\end{enumerate}
\end{proof}

\section*{1.7.19}

\subsection*{Bijection}

\begin{proof}
	Proof of Injective by contradiction

	Suppose $\exists h_1,h_2 : h_1 \neq h_2 \land h_1x\equiv h_2x$,
	because $\exists x^{-1} : h_1x x^{-1} = h_2x x^{-1} = h_1 = h_2$, which is a contradtion.

	Proof of surjective

	There's nothing to be proved here because by definition $\forall o \in \mathcal{O} : \exists h : hx = o$.

\end{proof}

\subsection*{Lagrange's Theorem}

\begin{proof}
	From the preceding exercise we know that by applying $h$ to the element, we can define an equivalence relation.

	Therefore, we can see that $\mathcal{O}_x$ will define a partition of $G$.

	Further we know that $\forall x,y \in G \land \mathcal{O}_x \neq \mathcal{O}_y: |\mathcal{O}_x|=|\mathcal{O}_y|=|H|$.

	Because $\mathcal{O}$ is a partition, so $|G|=\sum\limits_{x}^{} \mathcal{O}_x$,
	combining the previous two statement, $\exists u \in \Z : |G| = u|\mathcal{O}| \implies |G|=u|H|$.
\end{proof}

\section*{2.1.3}
\subsection*{a}

\begin{proof}
	Check closed under inversion.

	\[
		r^4 = \one \implies r^2r^2 = \one \implies r^2 = {(r^2)}^{-1}
	\]

	\[
		s^2 = \one \implies s=s^{-1}
	\]

	\[
		(sr^2)^2 = sr^2sr^2 = sr^2r^{-2}s=ss=\one \implies sr^2 = (sr^2)^{-1}
	\]

	Closed under multiplcation (skip some trivial cases)

	\[
		r^2s=r^2s = r^{-2} s = sr^2
	\]

	\[
		r^2 sr^2 = sr^{-2}r^2 = s
	\]

	\[
		sr^2r^2=s\one=s
	\]

	\[
		ssr^2 = r^2
	\]

	\[
		sr^2s=ssr^{-2}=r^{-2}=r^{2}
	\]
\end{proof}

\subsection*{b}

\begin{proof}
	Close under inversion:

	$r^2$ has been checked before

	\[
		srsr=srr^{-1}s=\one
	\]

	\[
		sr^3sr^3=sr^3r^{-3}s=\one
	\]

	Close under multiplcation (skip trivial)

	\[
		r^2 sr = r^2r^{-1}s=rs=sr^{-1}=sr^3
	\]

	\[
		r^2sr^3=r^{-1}s=sr
	\]

	\[
		sr^3r^2=sr^5=sr
	\]

	\[
		srsr^3=srr^{-3}s=sr^-2s=s^2r^2=r^2
	\]

	\[
		sr^3sr=sr^3r^{-1}s=sr^2s=ssr^{-2}=r^{-2}=r^2
	\]
\end{proof}

\end{document}





