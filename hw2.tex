\pdfcompresslevel=0
\pdfobjcompresslevel=0

%: CLASS FILE

\documentclass[11pt]{article}
%\documentclass[11pt,reqno]{amsart} % AMS article style with right equation numbering


%: PACKAGES

\usepackage{graphicx}

% sets margins
\usepackage[top=1in,bottom=1in,left=1in,right=1in]{geometry}

% gives more color definitions
\usepackage{xcolor}

% gives optional arguments to list environments
\usepackage{enumitem}

\usepackage{microtype}

% uncomment to begin paragraphs with an empty line rather than an indent
% \usepackage[parfill]{parskip}

\usepackage{amsmath, amsthm, amssymb, tikz, mathtools, array, mathrsfs, tensor}

% must be loaded after ams packages
% gives various integral symbols: see http://ctan.math.illinois.edu/macros/latex/contrib/esint/esint.pdf
\usepackage{esint}

% creates command \one for blackboard 1 to use for indicator functions
\usepackage{dsfont}
	\newcommand{\one}{\mathds{1}}

% \framebox{$ stuff $} will frame things in line.
% \Aboxed{$ stuff $} will frame things in align environment.
\usepackage{framed}

% always have this as the last package listed.
\usepackage{hyperref}
	\hypersetup{hypertexnames=false,colorlinks=true,linkcolor=blue,citecolor=black,urlcolor=blue}
	
	
%: SETTINGS

% uncomment to align environments to be split over page breaks
%\allowdisplaybreaks

%: COMMANDS

% equation shortcuts
\newcommand{\eq}[1]{\begin{align*} #1 \end{align*}}
\newcommand{\eeq}[1]{\begin{align} \begin{split} #1 \end{split} \end{align}}

% shortcut for stacking equation references over relations
% usage: \stackref{LABEL}{SYMBOL}
\newcommand{\stackref}[2]{\stackrel{\mbox{\footnotesize{\eqref{#1}}}}{#2}}
\newcommand{\stackrefp}[2]{\stackrel{\phantom{\mbox{\footnotesize{\eqref{#1}}}}}{#2}}

% variable Greek letters
\def\eps{\varepsilon}
\def\vphi{\varphi}

% blackboard letters
\newcommand{\E}{\mathbb{E}}
\newcommand{\F}{\mathbb{F}}
\newcommand{\N}{\mathbb{N}}
\renewcommand{\P}{\mathbb{P}}
\newcommand{\Q}{\mathbb{Q}}
\newcommand{\R}{\mathbb{R}}
\renewcommand{\S}{\mathbb{S}}
\newcommand{\T}{\mathbb{T}}
\newcommand{\Z}{\mathbb{Z}}

% calligraphic letters
\renewcommand{\AA}{\mathcal{A}}
\newcommand{\BB}{\mathcal{B}}
\newcommand{\CC}{\mathcal{C}}
\newcommand{\DD}{\mathcal{D}}
\newcommand{\EE}{\mathcal{E}}
\newcommand{\FF}{\mathcal{F}}
\newcommand{\GG}{\mathcal{G}}
\newcommand{\HH}{\mathcal{H}}
\newcommand{\II}{\mathcal{I}}
\newcommand{\JJ}{\mathcal{J}}
\newcommand{\KK}{\mathcal{K}}
\newcommand{\LL}{\mathcal{L}}
\newcommand{\MM}{\mathcal{M}}
\newcommand{\NN}{\mathcal{N}}
\newcommand{\OO}{\mathcal{O}}
\newcommand{\PP}{\mathcal{P}}
\newcommand{\QQ}{\mathcal{Q}}
\newcommand{\RR}{\mathcal{R}}
\renewcommand{\SS}{\mathcal{S}}
\newcommand{\TT}{\mathcal{T}}
\newcommand{\UU}{\mathcal{U}}
\newcommand{\VV}{\mathcal{V}}
\newcommand{\WW}{\mathcal{W}}
\newcommand{\XX}{\mathcal{X}}
\newcommand{\YY}{\mathcal{Y}}
\newcommand{\ZZ}{\mathcal{Z}}

% script letters
\newcommand{\AAA}{\mathcal{A}}
\newcommand{\BBB}{\mathcal{B}}
\newcommand{\CCC}{\mathcal{C}}
\newcommand{\DDD}{\mathcal{D}}
\newcommand{\EEE}{\mathcal{E}}
\newcommand{\FFF}{\mathcal{F}}
\newcommand{\GGG}{\mathcal{G}}
\newcommand{\HHH}{\mathcal{H}}
\newcommand{\III}{\mathcal{I}}
\newcommand{\JJJ}{\mathcal{J}}
\newcommand{\KKK}{\mathcal{K}}
\newcommand{\LLL}{\mathcal{L}}
\newcommand{\MMM}{\mathcal{M}}
\newcommand{\NNN}{\mathcal{N}}
\newcommand{\OOO}{\mathcal{O}}
\newcommand{\PPP}{\mathcal{P}}
\newcommand{\QQQ}{\mathcal{Q}}
\newcommand{\RRR}{\mathcal{R}}
\newcommand{\SSS}{\mathcal{S}}
\newcommand{\TTT}{\mathcal{T}}
\newcommand{\UUU}{\mathcal{U}}
\newcommand{\VVV}{\mathcal{V}}
\newcommand{\WWW}{\mathcal{W}}
\newcommand{\XXX}{\mathcal{X}}
\newcommand{\YYY}{\mathcal{Y}}
\newcommand{\ZZZ}{\mathcal{Z}}

% inner product 
% usage: \iprod{left}{right}
\newcommand{\iprod}[2]{\langle #1,\, #2\rangle}

% bold math font (still in italics) 
% usage: {\vc CONTENT}
\newcommand{\vc}[1]{{\boldsymbol #1}}

% widetilde and widehat shortcuts
\newcommand{\wt}[1]{\widetilde{#1}}
\newcommand{\wh}[1]{\widehat{#1}}

% math operators with subcripts
\DeclareMathOperator{\Var}{Var}
\DeclareMathOperator{\Cov}{Cov}
\DeclareMathOperator{\Ent}{Ent}

% math operators with subtext
\DeclareMathOperator*{\esssup}{ess\,sup}
\DeclareMathOperator*{\essinf}{ess\,inf}

% conditional statements (size of delimiter as an optional command)
\newcommand{\givenc}[3][]{#1\{ #2 \: #1| \: #3 #1\}} % with braces
\newcommand{\givenk}[3][]{#1[ #2 \: #1| \: #3 #1]} % with brackets
\newcommand{\givenp}[3][]{#1( #2 \: #1| \: #3 #1)} % with parentheses
\newcommand{\givena}[3][]{#1\langle #2 \: #1| \: #3 #1\rangle} % with angle brackets

% common math text
\DeclareMathOperator{\e}{e} % exponential
\newcommand{\cc}{\mathrm{c}} % for set complements and "critical" subscripts
\newcommand{\dd}{\mathrm{d}} % for differentials
\newcommand{\Exp}{\mathrm{Exp}} % for exponential distribution

% floor and ceiling functions
\DeclarePairedDelimiter\ceil{\lceil}{\rceil}
\DeclarePairedDelimiter\floor{\lfloor}{\rfloor}

%: ENVIRONMENTS

% exercise environment
\newenvironment{exercise}[1]
	{\noindent \textbf{#1:}}
	{\par \vspace{0.5\baselineskip}}
	
% solution environment
\newenvironment{solution}[1][\unskip]
	{\noindent \textbf{Solution #1:} }
	{\qed \pagebreak}
	
\newenvironment{solutionn}[1][\unskip]
	{\noindent \textbf{Solution #1:} }
	{\qed \\}	

%: TITLE

\begin{document}
\bibliographystyle{acm}

\begin{center}
	\framebox{\parbox{\linewidth}{\centering
	{\bf{Homework 2}}\\
	MATH 541: Abstract Algebra 1\\
	Spring 2023 \\[\baselineskip]
	{\sc Hongtao Zhang}}} %at \href{mailto:ewbates@wisc.edu}{\nolinkurl{ewbates@wisc.edu}}.}}
\end{center}

Sec. 1.4: 10

Sec. 1.6: 14, 18, 24, 25(a)(b)

Sec. 1.7: 16, 17
%: BODY

\section*{1.4}

\subsection*{10}

\begin{enumerate}
	\item \begin{proof}
		      $$
			      \begin{pmatrix}
				      a_1 & b_1 \\
				      0   & c_1
			      \end{pmatrix}
			      \begin{pmatrix}
				      a_2 & b_2 \\
				      0   & c_2
			      \end{pmatrix} =
			      \begin{pmatrix}
				      a_1a_2 & a_1b_2+b_1c_2 \\
				      0      & c_1c_2
			      \end{pmatrix}
		      $$

		      Because $a_1, a_2 \neq 0$, so $a_1a_2\neq 0$, same for $c_1,c_2$.
		      Therefore, G is closed under matrix mul.
	      \end{proof}
	\item \begin{proof}
		      $$
			      \begin{pmatrix}
				      a & b \\
				      0 & c
			      \end{pmatrix}^{-1} =
			      \begin{pmatrix}
				      1/a & -b/(ac) \\
				      0   & 1/c
			      \end{pmatrix}
		      $$
	      \end{proof}

	      Because $a,c\neq 0$, so all entries are well defined within $\R$, which means it is closed.

	\item Because any matrix operation defined in $GL_2(\R)$ is defined in $G$,
	      and $G$ is clearly closed under $addition$ and $subtraction$,
	      and $G$ is a subset of $GL_2(\R)$ where the left lower entry is $0$, and $a,c \neq 0$,
	      so $G$ is a subgroup of $GL_2(\R)$.
	\item Follow the similar steps from above, it suffices to check whether the new set $G'$ is closed under multiplication and inverse.
	      \begin{proof}
		      $$
			      \begin{pmatrix}
				      a_1 & b_1 \\
				      0   & a_1
			      \end{pmatrix}
			      \begin{pmatrix}
				      a_2 & b_2 \\
				      0   & a_2
			      \end{pmatrix} =
			      \begin{pmatrix}
				      a_1a_2 & a_1b_2+a_2b_1 \\
				      0      & a_1a_2
			      \end{pmatrix}
		      $$
		      The left top entry and the right bottom entry are the same,
		      which indicates that matrix multiplication is closed in $G'$.

		      $$
			      \begin{pmatrix}
				      a & b \\
				      0 & a
			      \end{pmatrix}^{-1}=
			      \begin{pmatrix}
				      \frac{1}{a} & -\frac{b}{a^2} \\
				      0           & \frac{1}{a}
			      \end{pmatrix}
		      $$

		      which is also inside $G'$, therefore $G'$ is a subgroup of $G$.

		      Therefore, $G'$ is a subgroup of $GL_2(\R)$.
	      \end{proof}
\end{enumerate}

\section*{1.6}

\subsection*{14}

\subsubsection*{Kernel is a subgroup}

\begin{proof}
	Denote operation on $G$ as $\star_G$,
	mutatis mutandis for $\star_H$.

	Consider two element $x,y \in kernel(H)$.

	$$
		\phi(x \star_G y) = \phi(x) \star_H \phi(y) = \one \star_H \one = \one
	$$

	Therefore $x \star_G y$ is also in $kernel(H)$.

	$$
		\phi(x \star_G x^{-1}) = \phi(x) \star_H \phi(x^{-1}) = \one_H \star_H \phi(x^{-1}) = \one_H \implies \phi(x^{-1}) = \one_H
	$$

	Therefore $x^{-1}$ is also in $kernel(H)$.

	Therefore, $kernel(H)$ is a subgroup of $G$.
\end{proof}

\subsubsection*{injective iff kernel is the identity subgroup of G}

\begin{proof}
	First prove that $\phi$ is injective if the kernel of $\phi$ is the identity subgroup of $G$.

	$$
		kernel(H) = \one_G
	$$

	Assume $\phi$ is not injective, i.e. there exists two element $a,b \in G$ that $\phi(a) = \phi(b)$ but $a\neq b$.
	$$
		\phi(a \star a^{-1} \star b) = \phi(b) = \phi(a) = \phi(a) \star \phi(a^{-1} \star b) \implies \phi(a^{-1} \star b) = \one
	$$

	However, we know that only $\phi(\one) = \one$, but $a\neq b$, so $a^{-1} \star b \neq \one$, which is a contradiction.

	Then prove If $\phi$ is injective, then the kernel is the identity subgroup.

	We know that the identity subgroup of G always map to the identity subgroup of H,
	so by injectivity, it is the only subgroup lies in the kernel.


\end{proof}

\subsection*{18}

\begin{proof}
	If $G$ is abelian, then $\forall a,b \in G: a \star b = b \star a$.

	Denote the map as $\phi$

	$$
		\phi(a \star b) = (a \star b) \star (a \star b) = a \star a \star b \star b = \phi(a) \star \phi(b)
	$$

	If $\phi(a \star b) = \phi(a) \star \phi(b)$
	$$
		(a \star b)\star(a \star b) = a \star a \star b \star b \implies b \star a = a \star b
	$$, which means $\star$ is commutative under $G$.

\end{proof}

\subsection*{24}

We can write G as
$$
	G = \left\{ x, y, xy, yx, (xy)^2, (yx)^2 , ... \right\}
$$

We can show that $yx$ also have order $n$

$$
	(yx)^{n+1} = y(xy)^nx=yx \implies (yx)^n = \one
$$

Therefore there's $n-1$ elements that is power of $xy$,
$n-1$ elements that is power of $yx$, and $x,y$, so the over all $|G|=2n$.

We have proved in the last homework that $D_{2n}$ can be generated by
$s$ and $sr$, which both have order $2$.

Therefore, if we construct a mapping from $G\to D_{2n}$ that maps $x$ to $s$,
and $y$ to $sr$, it is a isomorphism.

\subsection*{25}

\begin{enumerate}
	\item \begin{proof}
		      $$
			      \begin{pmatrix}
				      \cos (\theta) & -\sin(\theta) \\
				      \sin (\theta) & \cos(\theta)
			      \end{pmatrix}
			      \begin{pmatrix}
				      x \\
				      y
			      \end{pmatrix} =
			      \begin{pmatrix}
				      x \cos \theta - y \sin \theta \\
				      x \sin \theta + y \cos \theta
			      \end{pmatrix}
		      $$

		      It suffics to check the two basis.

		      For $(1,0)$, after applying the matrix,
		      it becomes $(\cos \theta, \sin \theta)$, which is true by definition.

		      For $(0,1)$, after applying the matrix,
		      it becomes $(-\sin \theta, \cos \theta)$, which is true by rotating the axis by 90 degree.

	      \end{proof}
	\item We know that $\theta=\frac{2\pi}{n}$ so
	      $$
		      \phi(r^n) = \phi(r)^n = \begin{pmatrix}
			      \cos (\theta) & -\sin(\theta) \\
			      \sin (\theta) & \cos(\theta)
		      \end{pmatrix}^n = \one
	      $$
	      $$
		      \phi(s^2) = \begin{pmatrix}
			      0 & 1 \\
			      1 & 0
		      \end{pmatrix}^2 = \one
	      $$

	      \begin{align*}
		      \phi(rs) & = \phi(r) \star \phi(s)                      \\
		               & = \begin{pmatrix}
			                   \cos (\theta) & -\sin(\theta) \\
			                   \sin (\theta) & \cos(\theta)
		                   \end{pmatrix}\begin{pmatrix}
			                                0 & 1 \\
			                                1 & 0
		                                \end{pmatrix}              \\
		               & = \begin{pmatrix}
			                   -\sin(\theta) & \cos (\theta) \\
			                   \cos(\theta)  & \sg \cdot ain (\theta)
		                   \end{pmatrix}              \\
		               & = \begin{pmatrix}
			                   0 & 1 \\
			                   1 & 0
		                   \end{pmatrix}\begin{pmatrix}
			                                \cos (\theta)  & \sin(\theta) \\
			                                -\sin (\theta) & \cos(\theta)
		                                \end{pmatrix} \\
		               & = \begin{pmatrix}
			                   0 & 1 \\
			                   1 & 0
		                   \end{pmatrix}\begin{pmatrix}
			                                \cos (\theta) & -\sin(\theta) \\
			                                \sin (\theta) & \cos(\theta)
		                                \end{pmatrix}^{-1} \\
		               & = \phi(sr^{-1})
	      \end{align*}
\end{enumerate}

\section*{1.7}

\subsection*{16}

\begin{proof}
	\begin{enumerate}
		\item $(gg) \cdot a = ggag^{-2} = g \cdot (gag^{-1}) = g \cdot (g \cdot a)$
		\item $\one \cdot a = \one a \one^{-1} = a$
	\end{enumerate}
\end{proof}


\subsection*{17}

\begin{proof}
	We can find the inverse of the mapping easily that is simply $x \mapsto g^{-1}xg$, which means it is bijective.

	Assume $x^n=\one$, so $(gxg^{-1})^n = gx^ng^{-1} = \one$.
	Also because $|x|=n$, so any power less than $n$ is not identity.

	We know that the mapping is a isomorphic mapping, so it is injective.
	Therefore, it is clear that $|A|=|gAg^{-1}|$.
	
\end{proof}


\end{document}





